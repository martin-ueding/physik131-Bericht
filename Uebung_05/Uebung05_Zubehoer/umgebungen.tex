%
% Quellcode zu 'umebungen.tex'
%
\documentclass[a4paper]{article}
\usepackage[ngerman]{babel}
%
\begin{document}
\title{Mathematische Formelsammlung \\
(Zur Demonstration der mathematischen Umgebungen)}
\author{Thomas Erben, Klaus Lehnertz,Oliver Cordes}
\maketitle

%
% Der folgende Text demonstriert die verschiedenen mathematischen Modi.
% - $ ... $ fuer Formeln innerhalb des laufenden Textes
% - \[ ... \] fuer vom Text abgesetzte Formeln
% - \begin{equation} ... \end{equation} fuer vom Text abgesetzte Formeln
%   mit fortlaufender Numerierung
% - \begin{eqnarray} ... \end{eqnarray} fuer vom Text abgesetzte, mehrzeilige
%   Formeln mit fortlaufender Numerierung. Die einzelnen Zeilen sehen so aus:
%
%   linker Formelteil 1 & = & rechter Formelteil 1 \\
%   linker Formelteil 2 & = & rechter Formelteil 2 \nonumber \\
%
%   Ein '\nonumber' in einer der Zeilen heisst, dass diese Zeile nicht
%   numeriert wird!
\section{Die Summenformel von Gau"s}
Die Summe der Zahlen $1\ldots n$ berechnen wir mit einer bekannten Formel
von Gau"s mit $1+2+3+\ldots+n=\frac{n(n+1)}{2}$. Vom Text abgesetzt sieht 
das Ganze so aus:
%
\[
  1+2+3+\ldots+n=\frac{n(n+1)}{2}.
\]
%
Nat"urlich k"onnen wir Gleichungen auch Nummern geben:
%
\begin{equation}
  1+2+3+\ldots+n=\frac{n(n+1)}{2}.
\end{equation}
%
\section{Darstellungen der {\sc Euler}schen Zahl $e$}
Diese Gleichungen werden dann automatisch fortlaufend nummeriert, z. B. wenn wir
die \textsc{Euler}sche Zahl $e$ mit einer unendlichen Summe berechnen:
%
\begin{equation}
  e=\sum_{n=1}^{\infty}\frac{1}{n!}.
\end{equation}
%
Eine andere Darstellung dieser Zahl ist:
%
\begin{equation}
  e=\lim_{n\to\infty}\left(1+\frac{1}{n}\right)^{n}.
\end{equation}
%
\section{Die Binomischen Formeln}
Die drei binomischen Formeln lauten:
%
\begin{eqnarray}
  (a+b)^{2} & = & a^{2}+2ab+b^{2} \nonumber \\
  (a-b)^{2} & = & a^{2}-2ab+b^{2}  \\
  (a+b)(a-b) & = & a^{2}-b^{2} \nonumber
\end{eqnarray}
%
\end{document}

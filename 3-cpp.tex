% Copyright (c) 2011 Martin Ueding <dev@martin-ueding.de>

\part{C++}

\section{Heron}

Der Algorithmus ist recht einfach umzusetzen, man benötigt die Formel, die auf dem Übungsblatt angegeben ist.

\begin{equation}
x_{n+1}=\frac{x_n+\frac{a}{x_n}}{2}
\end{equation}

Dabei ist $a$ die Zahl, von der die Wurzel bestimmt werden soll. Diese Formel braucht noch ein $x_0$, damit sie funktioniert. Hier kann man einfach jede beliebige Zahl benutzen. Allerdings ist $x_0=a$ keine allzu schlechte Wahl.

Eine Abbruchbedingung kann verschieden aussehen. Man kann nach einer gewissen Anzahl von Iterationen abbrechen. Ich habe mich dafür entschieden, abzubrechen, sobald sich die Zahl nur noch wenig verändert. Dazu muss man sich merken, welchen Wert die Zahl vorher hatte, um vergleichen zu können.

Das ganze Programm könnte man rekursiv implementieren, allerdings spricht hier nichts gegen eine iterative Variante, daher habe ich mich für letzteres entschieden. Mein Programm ist in Listing \ref{code:heron} zu sehen.

Der Quellcode ist in Englisch dokumentiert, wie in \cite{complete} und \cite{rapid} empfohlen.

\code[c++]{Uebung_06/heron/heron.cpp}{heron.cpp}{code:heron}

Ich lasse $\sqrt{2} \approx 1.41421356237309504880168872421$\footnote{Mit \texttt{N[Sqrt[2], 30]} in Mathematica berechnet.} berechnen, die Ausgabe ist in Listing \ref{code:heron.2.out} zu sehen. Man kann sehen, dass das Programm schnell zu einem brauchbaren Ergebnis kommt, der Algorithmus also leistungsfähig ist.

\code{Uebung_06/heron/heron.2.out}{Ausgabe von heron für 2}{code:heron.2.out}
\code{Uebung_06/heron/heron.4.out}{Ausgabe von heron für 4}{code:heron.4.out}
\code{Uebung_06/heron/heron.10.out}{Ausgabe von heron für 10}{code:heron.10.out}

Für die CppDoc Kommentare habe ich in \cite{cppdoc} nachgeschaut.

\section{Berichtsaufgabe}

Das hier war noch auf den Aufgabenzetteln als Berichtsaufgabe gekennzeichnet, allerdings taucht es nicht in der finalen Liste auf:

Dieses Programm fügt mehrere Kommandozeilenargumente in eine Datei zusammen.

\code[c++]{Uebung_08/Consolidator/main.cpp}{Consolidator main.cpp}{code:consolidator-main.cpp}

\section{Versuchsergebnisse, die Zweite}

Dies ist das Programm aus der 7. Übung, entsprechend für die Berichtsaufgabe angepasst.

Bei der Bestimmung des Fehlers von $R$ weiß ich nicht genau, was $\delta R$ sein soll. Ich rechne jetzt mit

\begin{equation}
\Delta R = \sqrt{(\Delta U)^2 + (\Delta I)^2}
\end{equation}

\code[c++]{Uebung_09/bericht/bericht.cpp}{bericht.cpp}{}
\code{Uebung_09/bericht/bericht.out}{Ausgabedatei von bericht}{}
\code{Uebung_09/bericht/means.dat}{Ausgabe von bericht}{}

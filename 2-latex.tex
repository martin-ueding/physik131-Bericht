% Copyright (c) 2011 Martin Ueding <dev@martin-ueding.de>

\part{\LaTeX}

\chapter{Übung 4}

\textbf{Faust - Der Tragödie erster Teil, Johann Wolfgang von Goethe}

Quelle: \url{http://de.wikisource.org/}

\[ \cdots \]

\textbf{Faust.} \\
Das also war des Pudels Kern! \\
Ein fahrender Scolast? Der Casus macht mich lachen. \\

\begin{flushright}
\textbf{Mephistopheles.} \\
\textit{Ich salutire den gelehrten Herrn! \\
Ihr habt mich weidlich schwitzen machen.}
\end{flushright}

\textbf{Faust.} \\
Wie nennst du dich?

\begin{flushright}
\textbf{Mephistopheles.}
\textit{Die Frage scheint mir klein, \\
Für einen, der das Wort so sehr verachtet, \\
Der, weit entfernt von allem Schein, \\
Nur in der Wesen Tiefe trachtet.}
\end{flushright}

\textbf{Faust.} \\
Bey euch, ihr Herrn, kann man das Wesen \\
Gewöhnlich aus dem Namen lesen, \\
Wo es sich allzu deutlich weist, \\
Wenn man euch Fliegengott, Verderber, Lügner heit. \\
Nun gut wer bist du denn?

\begin{flushright}
\textbf{Mephistopheles.} \\
\textit{Ein Theil von jener Kraft, \\
Die stets das Böse will und stets das Gute schafft.}
\end{flushright}

\[ \cdots \]


\chapter{Übung 5}

Siehe \url{Uebung_05/mathe.pdf} für Matheübung.

\section{Einbindung von Graphiken}
Siehe \url{Uebung_05/Uebung05_Zubehoer/gesellschaft_vierte.pdf}.

\section{Weiteres zur Textstrukturierung}
Anstelle die Datei \texttt{gliederung.tex} zu formatieren, habe ich Verweise in diesen Bericht eingebracht, wo diese durchaus nützlich sein können.

Ein Anhang mit Code Listings ist auch schon vorhanden. Inzwischen habe ich die Anhänge in die jeweiligen Teile gepackt, somit habe ich momentan keine appendix-Umgebung am Ende.

Ein Inhaltsverzeichnis hat dieses Dokument natürlich auch.

Auf der Titelseite habe ich allerdings nur einen Autor. Würde man zwei haben wollen, kann man sie mit \verb#\\# trennen, so wie überall sonst auch.

Das Literaturverzeichnis habe ich diesem Artikel hinzugefügt.

\section{Berichtaufgaben}

Die Linuxbefehle (§\ref{commands}) habe ich als \texttt{itemize} gehalten, was ich übersichtlicher finde. Eine Tabelle kann als Tabelle \ref{table:compression-results} auf Seite \pageref{table:compression-results} gefunden werden. Diese Tabelle hat schon ein Label.

% Copyright (c) 2011 Martin Ueding <dev@martin-ueding.de>

\part{\LaTeX}

\chapter{Übung 4}

\textbf{Faust - Der Tragödie erster Teil, Johann Wolfgang von Goethe}

Quelle: \url{http://de.wikisource.org/}

\[ \cdots \]

\textbf{Faust.} \\
Das also war des Pudels Kern! \\
Ein fahrender Scolast? Der Casus macht mich lachen. \\

\begin{flushright}
\textbf{Mephistopheles.} \\
\textit{Ich salutire den gelehrten Herrn! \\
Ihr habt mich weidlich schwitzen machen.}
\end{flushright}

\textbf{Faust.} \\
Wie nennst du dich?

\begin{flushright}
\textbf{Mephistopheles.}
\textit{Die Frage scheint mir klein, \\
Für einen, der das Wort so sehr verachtet, \\
Der, weit entfernt von allem Schein, \\
Nur in der Wesen Tiefe trachtet.}
\end{flushright}

\textbf{Faust.} \\
Bey euch, ihr Herrn, kann man das Wesen \\
Gewöhnlich aus dem Namen lesen, \\
Wo es sich allzu deutlich weist, \\
Wenn man euch Fliegengott, Verderber, Lügner heit. \\
Nun gut wer bist du denn?

\begin{flushright}
\textbf{Mephistopheles.} \\
\textit{Ein Theil von jener Kraft, \\
Die stets das Böse will und stets das Gute schafft.}
\end{flushright}

\[ \cdots \]


\chapter{Übung 5}

\section{Mathematischer Formelsatz}

\subsection{Beispiele auf Aufgabenblatt}

\paragraph{Fließtext}
$a^2 + b^2 = c^2$

\paragraph{eckige Klammern}
\[ a^2 + b^2 = c^2 \]

\paragraph{equation Umgebung}
\begin{equation}
a^2 + b^2 = c^2
\end{equation}

%\paragraph{equation* Umbebung}
%\begin{equation*}
%a^2 + b^2 = c^2
%\end{equation*}

\subsection{matheuebung.pdf}

\subsubsection{Gleichungsumformungen}

\paragraph{Erstes Beispiel (Einfache Gleichung}
\begin{equation}
4x^2 + 2xv + v^2 = (2x + v)^2 - 2xv
\end{equation}

\paragraph{Zweites Beispiel (Gleichung über mehrere Zeilen)}
\begin{eqnarray}
(2x + 1)(2x - 1) &=& 7 \nonumber \\
4x^2 - 1 &=& 7 \nonumber \\
x^2 &=& 2  \nonumber\\
x &=& \pm 2
\end{eqnarray}
\paragraph{Drittes Beispiel (Hochstellung, Wurzeln)}
\begin{eqnarray}
\left( a^{\frac{p}{q}} \right)^{rq} &=& \left( \left( \sqrt[q]{a^p} \right)^q \right)^r \nonumber \\
&=& (a^p)^r = a^{rp}
\end{eqnarray}
\paragraph{Viertes Beispiel (Brüche)}
\begin{equation}
\frac{1-x^4}{(x^3)^2}-\left(\frac{1}{x}\right)^2=\frac{1-2x^4}{x^6}
\end{equation}

\paragraph{Fünftes Beispiel (Mathe im Fließtext )}
\begin{equation}
m=\frac{v^2r}{G}
\end{equation}
Mit $G = \unitfrac[6{,}67 \e{-11}]{Nm^2}{kg^2}$, $v = \unitfrac[29{,}77]{km}{s}$ und $r = \unit[1{,}49570 \e{8}]{km}$ ergibt
sich für die Masse $M$ der Sonne:
\begin{equation}
M = \frac
	{\left(\unitfrac[22{,}77 \e{3}]{m}{s} \right)^2 \cdot \unit[1{,}49570 \e{11}]{m}}
	{\unitfrac[6{,}67 \e{-11}]{Nm^2}{kg^2}}
= \unit[1{,}98 \e{30}]{kg}
\end{equation}

\paragraph{Sechstes Beispiel (Klammerausdruck, Array)}
\begin{eqnarray}
\ln(1+|u|) &=& x-c \nonumber \\
1+|u| &=& e^{x-c} \nonumber \\
|u| &=& e^{x-c}-1 \nonumber \\
u(x) &=& \begin{cases}
e^{x-c}-1 & \text{für} \, x > c \\
0 & \text{für}\, x=c \\
-e^{x-c}+1 & \text{für} \, x < c
\end{cases}
\end{eqnarray}

\paragraph{Siebtes Beispiel (Funktionen, verschachtelte Brüche)}
Aus der l’Hos\-pi\-tal\-schen Regel folgt:

\[
\lim_{x \rightarrow \infty} \frac{\ln\sin(\pi x)}{\ln\sin(x)} =
\lim_{x \rightarrow \infty} \frac
	{\pi \frac{\cos(\pi x)}{\sin(\pi x)}}
	{\frac{\cos(x)}{\sin(x)}} =
\lim_{x \rightarrow \infty} \frac{\pi \tan(x)}{\tan(\pi x)} =
\lim_{x \rightarrow \infty} \frac{\pi / \cos^2(x)}{\pi / \cos^2(\pi x)} =
\lim_{x \rightarrow \infty} \frac{\cos^2(\pi x)}{\cos^2(x)} =
1
\]

\subsubsection{Die Lösung von Integralen (Integrale)}
\begin{equation}
f(x)=\int_a^b \frac{3x^2}{x^3-1} \mathrm dx
\end{equation}

Exkurs:

\begin{align}
u &:= x^3-1 \nonumber \\
\frac{\mathrm dx}{\mathrm dy} &= 3x^2 \nonumber \\
\mathrm dx &= \frac{\mathrm du}{3x^2} \nonumber \\
\int \frac{3x^2}{x^3-1} \mathrm dx &= \int \frac{1}{u} \mathrm du \\
f(x) &= \ln(|u|) + c
\end{align}

Es folgt damit:

\[ f(x) = [\ln(x^3-1)]_a^b \]

\paragraph{Die Integral - Multiplikationsregel}

Es gilt:

\[ \int_a^b g'(x)f(x)\mathrm dx = [g(x)f(x)]_a^b - \int_a^b g(x)f'(x)\mathrm dx \]

Berechnen wir damit als Beispiel das Integral $\int \ln(x) \mathrm dx = \int 1 \cdot \ln(x) \mathrm dx$!
\textbf{Führen Sie die Berechnung des Integrals in diesem Dokument zu Ende!}\footnote{Falls dies an mich gerichtet sein sollte, beißt sich das mit der Aufgabenstellung, dass man das Dokument möglichst wie im Original nachbauen soll.}

\subsection{Vektoren und Matrizen (Vektoren, Arrays, Fortsetzungspunkte)}

Der Winkel $\alpha$ zwischen zwei Vektoren $\vec{a}$ und $\vec{b}$ ist gegen durch:

\[ \cos(\alpha) = \frac{\vec{a} \cdot \vec{b}}{|\vec{a}| \cdot |\vec{b}|} \]

Ein lineares Gleichungssystem $A \cdot x = b$, wobei $A = (A_{ij})_{n \times n}$ eine $n \times n$ Matrix und $x=(x_i)_n$ und $b=(b_i)_n$ Vektoren mit $n$ Elementen sind, sieht ausgeschrieben so aus:

\[
\left(
\begin{matrix}
a_{11} & a_{12} & \cdots & a_{1n} \\
a_{21} & a_{22} & \cdots & a_{2n} \\
\vdots & \vdots & \ddots & \vdots \\
a_{n1} & a_{n2} & \cdots & a_{nn} \\
\end{matrix}
\right)
\left(
\begin{matrix}
a_{x_1}  \\
a_{x_2}  \\
\vdots \\
a_{x_n}  \\
\end{matrix}
\right)
=
\left(
\begin{matrix}
a_{b_1}  \\
a_{b_2}  \\
\vdots \\
a_{b_n}  \\
\end{matrix}
\right)
\]

Die Summenschreibweise für die $i$-te Zeile dieser Matrix ist dann:

\[ \sum_{m=1}^n a_{im} \cdot x_m = b_i \]

\section{Einbindung von Graphiken}
Siehe \url{Uebung_05/Uebung05_Zubehoer/gesellschaft_vierte.pdf}.

\section{Weiteres zur Textstrukturierung}
Anstelle die Datei \texttt{gliederung.tex} zu formatieren, habe ich Verweise in diesen Bericht eingebracht, wo diese durchaus nützlich sein können.

Ein Anhang mit Code Listings ist auch schon vorhanden. Inzwischen habe ich die Anhänge in die jeweiligen Teile gepackt, somit habe ich momentan keine appendix-Umgebung am Ende.

Ein Inhaltsverzeichnis hat dieses Dokument natürlich auch.

Auf der Titelseite habe ich allerdings nur einen Autor. Würde man zwei haben wollen, kann man sie mit \verb#\\# trennen, so wie überall sonst auch.

Das Literaturverzeichnis habe ich diesem Artikel hinzugefügt.

\section{Berichtaufgaben}

Die Linuxbefehle (§\ref{commands}) habe ich als \texttt{itemize} gehalten, was ich übersichtlicher finde. Eine Tabelle kann als Tabelle \ref{table:compression-results} auf Seite \pageref{table:compression-results} gefunden werden. Diese Tabelle hat schon ein Label.

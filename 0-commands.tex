% Copyright (c) 2011 Martin Ueding <dev@martin-ueding.de>

\section{Befehle in der Konsole}
\label{commands}

\subsection{Benutzerverwaltung und -rechte}
\begin{itemize}
\dd{chmod}{Ändert Dateirechte.}
\dd{hostname}{Gibt den Rechnernamen aus.}
\dd{last}{Letzte Anmeldungen aller Benutzer.}
\dd{ps}{Gibt eine Prozessliste aus. Dieses Programm ist nicht interaktiv und eignet sich beispielsweise für Logdateien.}
\dd{top}{Vorgänger von \texttt{htop}, eine interaktive Prozessverwaltung.}
\dd{uname}{Gibt Kernelversion, Rechnername, ..., aus.}
\dd{uptime}{Zeigt, wie viele Nächte der Rechner am Stück durchgemacht hat.}
\dd{whoami}{Gibt den eigenen Benutzernamen aus.}
\dd{w}{Wie \texttt{who}, nur ausführlicher.}
\end{itemize}

\subsection{Dateibehandlung}
\begin{itemize}
\dd{bzip2}{Komprimiert eine Datei.}
\dd{cd}{Wechselt in das angegebene Verzeichnis. Dabei ist \texttt{..} das übergeordnete Verzeichnis. Wird als Verzeichnis \texttt{-} angegeben, kommt man in das vorherige Verzeichnis. \texttt{cd} ohne Verzeichnis wechselt in das Heimatverzeichnis\footnote{Meistens \texttt{/home/<benutzername>}.}.}
\dd{cp}{Kopiert Dateien.}
\dd{df}{Listet die Dateisysteme mit Belegungsangabe.}
\dd{du}{Zeigt die Größe von Dateien auf dem Datenträger an. Diese muss nicht unbedingt mit der Größe überstimmen, die \texttt{ls} anzeigt, da die Dateien in Blöcken organisiert sind.}
\dd{find}{Führt eine Dateisystemtraverse nach speziellen Suchvorgaben durch und zeigt standardmäßig alle Dateien und Ordner an.}
\dd{gzip}{Komprimiert eine Datei.}
\dd{ls}{Listet den Inhalt des angegebenen oder aktuellen Verzeichnisses auf. Dabei werden auch Informationen über Zugriffsrechte und Eigentümer, Größe und Änderungsdatum angezeigt, gibt man \texttt{-l} an.}
\dd{mkdir}{Erstellt ein Verzeichnis.}
\dd{mv}{Verschiebt Dateien, Umbenennen ist ein Spezialfall.}
\dd{rmdir}{Löscht leere Verzeichnisse.}
\dd{rm}{Löscht Dateien.}
\dd{scp}{Kopiert über SSH.}
\dd{tar}{Erstellt ein Archiv mehrerer Dateien. Mit \texttt{tar -xzf archiv.tar.gt datei1 datei2…}\cite{man-tar} erstellt man direkt ein komprimiertes Archiv mehrerer Dateien.}
\dd{touch}{Setzt das Änderungsdatum einer Datei auf den aktuellen Zeitpunkt und erzeugt die Datei, falls sie nicht existiert.}
\end{itemize}

\subsection{Informationen zu Programmen}
\begin{itemize}
\dd{apropos}{Unscharfe Suche nach Befehlen.}
\dd{man}{Zeigt Handbücher zu Programmen an.}
\dd{whatis}{Zeigt die erste Zeile des Handbuchs an.}
\end{itemize}

\subsection{Textdateien}
\begin{itemize}
\dd{diff}{Vergleicht Dateien miteinander.}
\dd{emacs}{Texteditor.}
\dd{grep}{Filtert Zeilen nach einem Suchmuster.}
\dd{head}{Zeigt die ersten n Zeilen einer Datei an.}
\dd{less}{Zeigt eine Datei an und erlaubt scrollen, suchen, springen. Es können viele Befehle aus \texttt{vi} (gg, G, /, j, k) benutzt werden.}
\dd{pdflatex}{Übersetzt ein \LaTeX\ Dokument in ein PDF.}
\dd{sort}{Sortiert Zeilen.}
\dd{tail}{Gegenstück zu \texttt{head}.}
\dd{uniq}{Spezialfall von \texttt{sort -u}}
\dd{vim}{Texteditor für Programmierer.}
\dd{wc}{Zählt Wörter, Zeilen, Buchstaben.}
\end{itemize}

\subsection{Bash built-ins}
\begin{itemize}
\dd{clear}{Fügt leere Zeilen ein, bis der Bildschirm leer ist.}
\dd{set}{Zeigt das Environment der Shell an.}
\dd{echo}{Gibt Text aus.}
\dd{history}{Zeigt die letzten Kommandos an.}
\end{itemize}

\subsection{Diverses}
\begin{itemize}
\dd{bc}{Einfacher Taschenrechner. Man sollte in der Bash allerdings nicht mit \texttt{\$(echo 5+4 | bc)} rechnen, sondern die neuen, von C übernommenen Funktionen zum direkten Rechnen, \texttt{\$((5+4))}, benutzen.}
\dd{blkid}{Zeigt die UUID der Partitionen an.}
\dd{e2label}{Zeigt und vergibt Partitionslabel.}
\dd{ssh}{Öffnete eine Shell auf einem anderen Computer.}
\dd{wget}{Lädt Dateien per http oder ftp.}
\dd{cal}{Zeigt einen Kalendermonat an.}
\dd{date}{Zeigt das Datum in einem gewählten Format an.}
\dd{gv}{Zeigt eine PDF oder PS Datei an. Normalerweise würde man okular (KDE) oder evince (Gnome) benutzen.}
\end{itemize}

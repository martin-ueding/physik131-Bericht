% Copyright (c) 2011 Martin Ueding <dev@martin-ueding.de>
%
\documentclass[12pt]{article}
\usepackage{geometry}
\geometry{a4paper}
\usepackage{graphicx}
\usepackage{amssymb}
\usepackage{epstopdf}
\usepackage[utf8]{inputenc}
\usepackage[activate]{pdfcprot}
\usepackage{setspace}
\usepackage[ngerman]{babel}
\usepackage[parfill]{parskip}
\usepackage{hyperref}
\usepackage{color}
\definecolor{darkblue}{rgb}{0,0,.5}
\hypersetup{pdftex=true, colorlinks=true, breaklinks=false, linkcolor=black, menucolor=black, pagecolor=black, urlcolor=darkblue}
\DeclareGraphicsRule{.tif}{png}{.png}{`convert #1 `dirname #1`/`basename #1 .tif`.png}

\newcommand\gqq[1]{\glqq #1\grqq}

\newcommand\dd[2]{\item[\texttt{#1}] #2}

\title{EDV für Physiker}
\author{Martin Ueding}

\begin{document}

\maketitle

\tableofcontents
\newpage

\section{Übung 1}

\subsection{Befehle in der Konsole}
\label{commands}

\begin{itemize}

\dd{mkdir <dir>}{Erstellt ein Verzeichnis \texttt{<dir>}.}
\dd{cd <dir>}{Wechselt in das Verzeichnis \texttt{<dir>}. Dabei ist \texttt{..} das übergeordnete Verzeichnis. Wird als Verzeichnis \texttt{-} angegeben, kommt man in das vorherige Verzeichnis. \texttt{cd} ohne Verzeichnis wechselt in das Heimatverzeichnis\footnote{\texttt{$\sim$} oder meistens \texttt{/home/<benutzername>}.}}
\dd{ls -l [<dir>]}{ Listet den Inhalt des Verzeichnises \texttt{<dir>} oder des aktuellen Verzeichnises aus. Dabei werden auch Informationen über Zugriffsrechte und Eigentümer, Größe und Änderungsdatum angezeigt.}
\dd{gv <datei>}{Zeigt die PDF oder PS Datei \texttt{<datei>} an.}
\dd{whoami}{Eigener Benutzername.}
\dd{w}{Wie \texttt{who}, nur ausführlicher.}
\dd{hostname}{Rechnername.}
\dd{uname -a}{Kernelversion, Rechnername, ...}
\dd{last}{Letzte logins aller Benutzer.}
\dd{uptime}{Zeigt, wie viele Nächte der Rechner am Stück durchgemacht hat.}
\dd{whatis}{Zeigt die erste Zeile der man page an.}
\dd{apropos}{Unscharfe suche nach Befehlen.}
\dd{top}{Vorgänger von \texttt{htop}, ein Taskmanager.}
\dd{cal}{Zeigt einen Kalendermonat an.}
\dd{date}{Zeigt das Datum in einem gewählten Format an.}
\dd{touch}{Setzt das Änderungsdatum einer Datei auf den aktuellen Zeitpunkt und erzeugt die Datei, falls sie nicht existiert.}
\dd{less}{Zeigt eine Datei an und erlaubt scrollen, suchen, springen. Es können viele Befehle aus \texttt{vi} (gg, G, /, j, k) benutzt werden.}
\end{itemize}

\subsection{absoluter und relativer Pfad}

Ein absoluter Pfad beginnt immer mit einem \texttt{/}, wie beispielsweise \\ \texttt{/home/mu/Dokumente/Studium/EDV/Bericht} oder \texttt{/dev/null}. Ein relativer Pfad bezieht sich immer auf ein aktuelles Arbeitsverzeichnis. Beispielsweise beschreibt \texttt{datei.tex} die Datei \texttt{/tmp/datei.txt}, falls der Benutzer gerade \texttt{/tmp} als Arbeitsverzeichnis hat. Pfade können auch \texttt{..} enthalten, dies bezeichnet das übergeordnete Verzeichnis. Ist man gerade in \texttt{/etc/apache2}, so kann man mit \texttt{../passwd} auf die zentrale Passwortdatei verweisen.

Gemeinerweise können absolute Pfade auch \texttt{..} enthalten, so wäre \\
\texttt{/etc/apache2/../passwd} ein legaler Pfad, sinnvoll ist es in vielen Fällen allerdings nicht.

\subsection{grundlegende Emacs Steuerung}

\begin{tabular}{lc}
Aktion & Tasten \\
\hline
Seite rauf & M-v \\
Seite runter & C-v \\
Cursor links & C-b \\
Cursor rechts & C-f \\
Cursor rauf & C-p \\
Cursor runter & C-n \\
Datei speichern & C-x C-s \\
Emacs beenden & C-x C-c \\
\end{tabular}

\end{document}
